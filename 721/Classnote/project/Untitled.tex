% Options for packages loaded elsewhere
% Options for packages loaded elsewhere
\PassOptionsToPackage{unicode}{hyperref}
\PassOptionsToPackage{hyphens}{url}
\PassOptionsToPackage{dvipsnames,svgnames,x11names}{xcolor}
%
\documentclass[
  letterpaper,
  DIV=11,
  numbers=noendperiod]{scrartcl}
\usepackage{xcolor}
\usepackage{amsmath,amssymb}
\setcounter{secnumdepth}{-\maxdimen} % remove section numbering
\usepackage{iftex}
\ifPDFTeX
  \usepackage[T1]{fontenc}
  \usepackage[utf8]{inputenc}
  \usepackage{textcomp} % provide euro and other symbols
\else % if luatex or xetex
  \usepackage{unicode-math} % this also loads fontspec
  \defaultfontfeatures{Scale=MatchLowercase}
  \defaultfontfeatures[\rmfamily]{Ligatures=TeX,Scale=1}
\fi
\usepackage{lmodern}
\ifPDFTeX\else
  % xetex/luatex font selection
\fi
% Use upquote if available, for straight quotes in verbatim environments
\IfFileExists{upquote.sty}{\usepackage{upquote}}{}
\IfFileExists{microtype.sty}{% use microtype if available
  \usepackage[]{microtype}
  \UseMicrotypeSet[protrusion]{basicmath} % disable protrusion for tt fonts
}{}
\makeatletter
\@ifundefined{KOMAClassName}{% if non-KOMA class
  \IfFileExists{parskip.sty}{%
    \usepackage{parskip}
  }{% else
    \setlength{\parindent}{0pt}
    \setlength{\parskip}{6pt plus 2pt minus 1pt}}
}{% if KOMA class
  \KOMAoptions{parskip=half}}
\makeatother
% Make \paragraph and \subparagraph free-standing
\makeatletter
\ifx\paragraph\undefined\else
  \let\oldparagraph\paragraph
  \renewcommand{\paragraph}{
    \@ifstar
      \xxxParagraphStar
      \xxxParagraphNoStar
  }
  \newcommand{\xxxParagraphStar}[1]{\oldparagraph*{#1}\mbox{}}
  \newcommand{\xxxParagraphNoStar}[1]{\oldparagraph{#1}\mbox{}}
\fi
\ifx\subparagraph\undefined\else
  \let\oldsubparagraph\subparagraph
  \renewcommand{\subparagraph}{
    \@ifstar
      \xxxSubParagraphStar
      \xxxSubParagraphNoStar
  }
  \newcommand{\xxxSubParagraphStar}[1]{\oldsubparagraph*{#1}\mbox{}}
  \newcommand{\xxxSubParagraphNoStar}[1]{\oldsubparagraph{#1}\mbox{}}
\fi
\makeatother


\usepackage{longtable,booktabs,array}
\usepackage{calc} % for calculating minipage widths
% Correct order of tables after \paragraph or \subparagraph
\usepackage{etoolbox}
\makeatletter
\patchcmd\longtable{\par}{\if@noskipsec\mbox{}\fi\par}{}{}
\makeatother
% Allow footnotes in longtable head/foot
\IfFileExists{footnotehyper.sty}{\usepackage{footnotehyper}}{\usepackage{footnote}}
\makesavenoteenv{longtable}
\usepackage{graphicx}
\makeatletter
\newsavebox\pandoc@box
\newcommand*\pandocbounded[1]{% scales image to fit in text height/width
  \sbox\pandoc@box{#1}%
  \Gscale@div\@tempa{\textheight}{\dimexpr\ht\pandoc@box+\dp\pandoc@box\relax}%
  \Gscale@div\@tempb{\linewidth}{\wd\pandoc@box}%
  \ifdim\@tempb\p@<\@tempa\p@\let\@tempa\@tempb\fi% select the smaller of both
  \ifdim\@tempa\p@<\p@\scalebox{\@tempa}{\usebox\pandoc@box}%
  \else\usebox{\pandoc@box}%
  \fi%
}
% Set default figure placement to htbp
\def\fps@figure{htbp}
\makeatother





\setlength{\emergencystretch}{3em} % prevent overfull lines

\providecommand{\tightlist}{%
  \setlength{\itemsep}{0pt}\setlength{\parskip}{0pt}}



 


\usepackage{booktabs}
\usepackage{longtable}
\usepackage{array}
\usepackage{multirow}
\usepackage{wrapfig}
\usepackage{float}
\usepackage{colortbl}
\usepackage{pdflscape}
\usepackage{tabu}
\usepackage{threeparttable}
\usepackage{threeparttablex}
\usepackage[normalem]{ulem}
\usepackage{makecell}
\usepackage{xcolor}
\KOMAoption{captions}{tableheading}
\makeatletter
\@ifpackageloaded{caption}{}{\usepackage{caption}}
\AtBeginDocument{%
\ifdefined\contentsname
  \renewcommand*\contentsname{Table of contents}
\else
  \newcommand\contentsname{Table of contents}
\fi
\ifdefined\listfigurename
  \renewcommand*\listfigurename{List of Figures}
\else
  \newcommand\listfigurename{List of Figures}
\fi
\ifdefined\listtablename
  \renewcommand*\listtablename{List of Tables}
\else
  \newcommand\listtablename{List of Tables}
\fi
\ifdefined\figurename
  \renewcommand*\figurename{Figure}
\else
  \newcommand\figurename{Figure}
\fi
\ifdefined\tablename
  \renewcommand*\tablename{Table}
\else
  \newcommand\tablename{Table}
\fi
}
\@ifpackageloaded{float}{}{\usepackage{float}}
\floatstyle{ruled}
\@ifundefined{c@chapter}{\newfloat{codelisting}{h}{lop}}{\newfloat{codelisting}{h}{lop}[chapter]}
\floatname{codelisting}{Listing}
\newcommand*\listoflistings{\listof{codelisting}{List of Listings}}
\makeatother
\makeatletter
\makeatother
\makeatletter
\@ifpackageloaded{caption}{}{\usepackage{caption}}
\@ifpackageloaded{subcaption}{}{\usepackage{subcaption}}
\makeatother
\usepackage{bookmark}
\IfFileExists{xurl.sty}{\usepackage{xurl}}{} % add URL line breaks if available
\urlstyle{same}
\hypersetup{
  pdftitle={Project 1: Ovarian Cancer Analytic Dataset Preparation},
  pdfauthor={Jiaqi Wang},
  colorlinks=true,
  linkcolor={blue},
  filecolor={Maroon},
  citecolor={Blue},
  urlcolor={Blue},
  pdfcreator={LaTeX via pandoc}}


\title{Project 1: Ovarian Cancer Analytic Dataset Preparation}
\author{Jiaqi Wang}
\date{2025-10-10}
\begin{document}
\maketitle


\subsection{Introduction}\label{introduction}

Women with active ovarian cancer receive chemotherapy approximately
every two to three weeks.Physicians are concerned about patients
visiting the emergency department (ED) or being hospitalized between
chemotherapy appointments.The goal of this project is to \textbf{process
patient-level and encounter-level data} to create a clean, analytic
dataset that will support future modeling of unanticipated hospital
admissions (UHA).

\subsection{1. Data Import}\label{data-import}

Both datasets are imported without hard-coding file paths using the here
package.

\subsection{2. Merge the patient-level data into the encounter-level
data}\label{merge-the-patient-level-data-into-the-encounter-level-data}

After merging the patient-level and encounter-level datasets using
**MRN** as the unique identifier, ~the analytic dataset now contains all
encounter records with corresponding patient information. ~

Below is a brief preview showing the number of rows, variables, and the
first few records.

\begin{verbatim}
[1] 550
\end{verbatim}

\begin{verbatim}
[1] 550
\end{verbatim}

\begin{verbatim}
 [1] "MRN"            "contact_date"   "enc_type"       "temp"          
 [5] "distress_score" "WBC"            "BMI.r"          "DOB"           
 [9] "race"           "financialclass" "ethnicity"      "hypertension"  
[13] "CHF"            "diabetes"      
\end{verbatim}

\begin{longtable}[t]{lcccccccclcccc}
\caption{\label{tab:unnamed-chunk-4}Preview of Analytic Dataset (first 10 rows)}\\
\toprule
MRN & contact\_date & enc\_type & temp & distress\_score & WBC & BMI.r & DOB & race & financialclass & ethnicity & hypertension & CHF & diabetes\\
\midrule
HJ9754 & 2016-06-26 & Office visit & 97.91 & 2 & 15.12 & 28.33 & 1999-06-05 & White & Private & non-Hispanic & NA & NA & NA\\
GE5166 & 2016-08-08 & Office visit & 99.03 & 2 & 6.86 & 38.22 & 1993-09-16 & White & Private & non-Hispanic & NA & NA & NA\\
XV9573 & 2018-01-20 & Office visit & 99.15 & 2 & 5.48 & 32.13 & 1976-09-27 & White & Private & non-Hispanic & NA & NA & NA\\
CQ9338 & 2015-07-05 & Office visit & 99.09 & 3 & 15.11 & 25.09 & 1961-07-19 & Black & Medicare & non-Hispanic & NA & NA & NA\\
DH1301 & 2018-03-25 & Office visit & 99.18 & 3 & 3.40 & 33.41 & 1957-06-30 & Other & Private & non-Hispanic & NA & NA & NA\\
\addlinespace
WQ8508 & 2019-08-25 & Office visit & 97.61 & 1 & 5.04 & 21.30 & 1970-05-16 & White & Medicare & non-Hispanic & NA & NA & NA\\
XE4615 & 2017-06-20 & Office visit & 99.66 & 4 & 16.43 & 30.18 & 1997-02-11 & Black & Medicare & non-Hispanic & NA & NA & NA\\
IO6623 & 2015-08-10 & Office visit & 99.43 & 2 & 2.87 & 26.04 & 1985-05-07 & Other & Medicare & non-Hispanic & NA & NA & NA\\
JV9469 & 2014-04-11 & ED/Hospitalization & 98.32 & NA & NA & -999.00 & 1964-10-30 & White & Private & non-Hispanic & NA & NA & NA\\
NE9449 & 2019-02-15 & Office visit & 97.18 & 4 & 8.38 & 37.36 & 1966-07-20 & White & Private & non-Hispanic & NA & NA & NA\\
\bottomrule
\end{longtable}

\subsection{3. Analytic Dataset
Description}\label{analytic-dataset-description}

\begin{verbatim}
Granularity: One row represents one patient encounter.
\end{verbatim}

\begin{verbatim}
Number of encounters: 550 
\end{verbatim}

\begin{verbatim}
Number of variables: 14 
\end{verbatim}

\begin{verbatim}
Unique patients: 50 
\end{verbatim}

The analytic dataset was created by merging the \textbf{encounter-level
dataset} and the \textbf{patient-level dataset} using the variable
\emph{MRN} as a unique patient identifier.

Each row in this dataset represents a \textbf{single patient encounter},
which may correspond to an office visit, an emergency department (ED)
visit, or a hospitalization.

The analytic dataset contains \textbf{550 encounters} from \textbf{50
unique patients} and includes \textbf{14 variables} in total.

The encounter-level variables capture clinical and visit-specific
information such as contact date, encounter type, temperature, distress
score, white blood cell count (WBC), and body mass index (BMI).

The patient-level variables include demographic characteristics (date of
birth, race, ethnicity, and financial class) and comorbid conditions
such as hypertension, congestive heart failure (CHF), and diabetes.

Together, these variables provide both longitudinal encounter data and
baseline patient characteristics, forming a clean and well-structured
analytic dataset that can be used to develop predictive models for
unanticipated hospital admissions (UHA) among ovarian cancer patients.

\subsection{4. Data cleaning}\label{data-cleaning}

Rules applied (from project notes):

\begin{itemize}
\item
  \textbf{DOB:} unrealistic birth years set to missing (e.g., year
  \textless{} 1910 → NA).
\item
  \textbf{BMI:} -999 is missing → recode to NA; then truncate to
  \textbf{10--50}.
\item
  \textbf{WBC:} values \textless{} 0.05 treated as detection-limit error
  → set to \textbf{0.05}; truncate values \textgreater{} 50 to
  \textbf{50}.
\item
  \textbf{Temperature / Distress:} constrained to plausible ranges
  (95--105 °F; 0--10).
\end{itemize}

After applying the cleaning rules above, we verified that implausible
values were corrected.\\
The table below compares the minimum and maximum values of key variables
before and after cleaning.

\begin{longtable}[t]{lcccc}
\caption{\label{tab:unnamed-chunk-6}Comparison of Selected Variables Before and After Data Cleaning}\\
\toprule
Variable & Before\_Min & Before\_Max & After\_Min & After\_Max\\
\midrule
WBC & 0.00 & 53.60 & 0.05 & 50.0\\
BMI.r & -999.00 & 352.13 & 10.00 & 50.0\\
temp & 96.26 & 103.20 & 96.26 & 103.2\\
distress\_score & 0.00 & 7.00 & 0.00 & 7.0\\
\bottomrule
\end{longtable}

As shown above, implausible or out-of-range values were truncated to
clinically reasonable limits, confirming that the dataset was
successfully cleaned.

\subsection{5. WBC Re-categorization}\label{wbc-re-categorization}

WBC is recategorized per assignment cut points.

\subsection{6. WBC Summary Table}\label{wbc-summary-table}

Counts and percentages of encounters within each WBC group.

\begin{longtable}[t]{lcc}
\caption{\label{tab:unnamed-chunk-8}Table A. Counts (%) of Encounters within Each WBC Category}\\
\toprule
WBC\_cat & Count & Percent\\
\midrule
Low (<3.2) & 169 & 30.7\\
Normal (3.2–9.8) & 196 & 35.6\\
High (>9.8) & 113 & 20.5\\
Not Taken & 72 & 13.1\\
\bottomrule
\end{longtable}

\subsection{7. Patient-Level Table 1}\label{patient-level-table-1}

Baseline characteristics at the \textbf{patient level} (race, ethnicity,
financial class, hypertension, CHF, diabetes).

\begin{table}

\caption{\label{tab:unnamed-chunk-9}Table 1. Patient-level Counts and Percentages (Baseline Characteristics)}
\centering
\begin{tabular}[t]{ll}
\toprule
  & Overall\\
\midrule
 & (N=50)\\
\addlinespace[0.3em]
\multicolumn{2}{l}{\textbf{race}}\\
\hspace{1em}Black & 10 (20.0\%)\\
\hspace{1em}Other & 3 (6.0\%)\\
\hspace{1em}White & 37 (74.0\%)\\
\addlinespace[0.3em]
\multicolumn{2}{l}{\textbf{ethnicity}}\\
\hspace{1em}Hispanic & 3 (6.0\%)\\
\hspace{1em}non-Hispanic & 47 (94.0\%)\\
\addlinespace[0.3em]
\multicolumn{2}{l}{\textbf{financialclass}}\\
\hspace{1em}Medicare & 29 (58.0\%)\\
\hspace{1em}Private & 21 (42.0\%)\\
\addlinespace[0.3em]
\multicolumn{2}{l}{\textbf{hypertension}}\\
\hspace{1em}No & 30 (60.0\%)\\
\hspace{1em}Yes & 20 (40.0\%)\\
\addlinespace[0.3em]
\multicolumn{2}{l}{\textbf{CHF}}\\
\hspace{1em}No & 45 (90.0\%)\\
\hspace{1em}Yes & 5 (10.0\%)\\
\addlinespace[0.3em]
\multicolumn{2}{l}{\textbf{diabetes}}\\
\hspace{1em}No & 48 (96.0\%)\\
\hspace{1em}Yes & 2 (4.0\%)\\
\bottomrule
\end{tabular}
\end{table}

\subsection{8.Brief Summary}\label{brief-summary}

The report produced a single analytic dataset with encounter-level rows,
merged with patient demographics and comorbidities.

Missing/implausible values were handled via explicit missing code
conversion and clinically guided truncation.

WBC was categorized into Low, Normal, High, and Not Taken, and required
summary tables were constructed.

This dataset is ready for downstream modeling of ED visits and
unanticipated hospital admissions.




\end{document}
